% Options for packages loaded elsewhere
\PassOptionsToPackage{unicode}{hyperref}
\PassOptionsToPackage{hyphens}{url}
%
\documentclass[
]{article}
\usepackage{amsmath,amssymb}
\usepackage{iftex}
\ifPDFTeX
  \usepackage[T1]{fontenc}
  \usepackage[utf8]{inputenc}
  \usepackage{textcomp} % provide euro and other symbols
\else % if luatex or xetex
  \usepackage{unicode-math} % this also loads fontspec
  \defaultfontfeatures{Scale=MatchLowercase}
  \defaultfontfeatures[\rmfamily]{Ligatures=TeX,Scale=1}
\fi
\usepackage{lmodern}
\ifPDFTeX\else
  % xetex/luatex font selection
\fi
% Use upquote if available, for straight quotes in verbatim environments
\IfFileExists{upquote.sty}{\usepackage{upquote}}{}
\IfFileExists{microtype.sty}{% use microtype if available
  \usepackage[]{microtype}
  \UseMicrotypeSet[protrusion]{basicmath} % disable protrusion for tt fonts
}{}
\makeatletter
\@ifundefined{KOMAClassName}{% if non-KOMA class
  \IfFileExists{parskip.sty}{%
    \usepackage{parskip}
  }{% else
    \setlength{\parindent}{0pt}
    \setlength{\parskip}{6pt plus 2pt minus 1pt}}
}{% if KOMA class
  \KOMAoptions{parskip=half}}
\makeatother
\usepackage{xcolor}
\usepackage[margin=2.54cm]{geometry}
\usepackage{color}
\usepackage{fancyvrb}
\newcommand{\VerbBar}{|}
\newcommand{\VERB}{\Verb[commandchars=\\\{\}]}
\DefineVerbatimEnvironment{Highlighting}{Verbatim}{commandchars=\\\{\}}
% Add ',fontsize=\small' for more characters per line
\usepackage{framed}
\definecolor{shadecolor}{RGB}{248,248,248}
\newenvironment{Shaded}{\begin{snugshade}}{\end{snugshade}}
\newcommand{\AlertTok}[1]{\textcolor[rgb]{0.94,0.16,0.16}{#1}}
\newcommand{\AnnotationTok}[1]{\textcolor[rgb]{0.56,0.35,0.01}{\textbf{\textit{#1}}}}
\newcommand{\AttributeTok}[1]{\textcolor[rgb]{0.13,0.29,0.53}{#1}}
\newcommand{\BaseNTok}[1]{\textcolor[rgb]{0.00,0.00,0.81}{#1}}
\newcommand{\BuiltInTok}[1]{#1}
\newcommand{\CharTok}[1]{\textcolor[rgb]{0.31,0.60,0.02}{#1}}
\newcommand{\CommentTok}[1]{\textcolor[rgb]{0.56,0.35,0.01}{\textit{#1}}}
\newcommand{\CommentVarTok}[1]{\textcolor[rgb]{0.56,0.35,0.01}{\textbf{\textit{#1}}}}
\newcommand{\ConstantTok}[1]{\textcolor[rgb]{0.56,0.35,0.01}{#1}}
\newcommand{\ControlFlowTok}[1]{\textcolor[rgb]{0.13,0.29,0.53}{\textbf{#1}}}
\newcommand{\DataTypeTok}[1]{\textcolor[rgb]{0.13,0.29,0.53}{#1}}
\newcommand{\DecValTok}[1]{\textcolor[rgb]{0.00,0.00,0.81}{#1}}
\newcommand{\DocumentationTok}[1]{\textcolor[rgb]{0.56,0.35,0.01}{\textbf{\textit{#1}}}}
\newcommand{\ErrorTok}[1]{\textcolor[rgb]{0.64,0.00,0.00}{\textbf{#1}}}
\newcommand{\ExtensionTok}[1]{#1}
\newcommand{\FloatTok}[1]{\textcolor[rgb]{0.00,0.00,0.81}{#1}}
\newcommand{\FunctionTok}[1]{\textcolor[rgb]{0.13,0.29,0.53}{\textbf{#1}}}
\newcommand{\ImportTok}[1]{#1}
\newcommand{\InformationTok}[1]{\textcolor[rgb]{0.56,0.35,0.01}{\textbf{\textit{#1}}}}
\newcommand{\KeywordTok}[1]{\textcolor[rgb]{0.13,0.29,0.53}{\textbf{#1}}}
\newcommand{\NormalTok}[1]{#1}
\newcommand{\OperatorTok}[1]{\textcolor[rgb]{0.81,0.36,0.00}{\textbf{#1}}}
\newcommand{\OtherTok}[1]{\textcolor[rgb]{0.56,0.35,0.01}{#1}}
\newcommand{\PreprocessorTok}[1]{\textcolor[rgb]{0.56,0.35,0.01}{\textit{#1}}}
\newcommand{\RegionMarkerTok}[1]{#1}
\newcommand{\SpecialCharTok}[1]{\textcolor[rgb]{0.81,0.36,0.00}{\textbf{#1}}}
\newcommand{\SpecialStringTok}[1]{\textcolor[rgb]{0.31,0.60,0.02}{#1}}
\newcommand{\StringTok}[1]{\textcolor[rgb]{0.31,0.60,0.02}{#1}}
\newcommand{\VariableTok}[1]{\textcolor[rgb]{0.00,0.00,0.00}{#1}}
\newcommand{\VerbatimStringTok}[1]{\textcolor[rgb]{0.31,0.60,0.02}{#1}}
\newcommand{\WarningTok}[1]{\textcolor[rgb]{0.56,0.35,0.01}{\textbf{\textit{#1}}}}
\usepackage{graphicx}
\makeatletter
\def\maxwidth{\ifdim\Gin@nat@width>\linewidth\linewidth\else\Gin@nat@width\fi}
\def\maxheight{\ifdim\Gin@nat@height>\textheight\textheight\else\Gin@nat@height\fi}
\makeatother
% Scale images if necessary, so that they will not overflow the page
% margins by default, and it is still possible to overwrite the defaults
% using explicit options in \includegraphics[width, height, ...]{}
\setkeys{Gin}{width=\maxwidth,height=\maxheight,keepaspectratio}
% Set default figure placement to htbp
\makeatletter
\def\fps@figure{htbp}
\makeatother
\setlength{\emergencystretch}{3em} % prevent overfull lines
\providecommand{\tightlist}{%
  \setlength{\itemsep}{0pt}\setlength{\parskip}{0pt}}
\setcounter{secnumdepth}{-\maxdimen} % remove section numbering
\ifLuaTeX
  \usepackage{selnolig}  % disable illegal ligatures
\fi
\IfFileExists{bookmark.sty}{\usepackage{bookmark}}{\usepackage{hyperref}}
\IfFileExists{xurl.sty}{\usepackage{xurl}}{} % add URL line breaks if available
\urlstyle{same}
\hypersetup{
  pdftitle={ENV 797 - Time Series Analysis for Energy and Environment Applications \textbar{} Spring 2024},
  pdfauthor={Student Name},
  hidelinks,
  pdfcreator={LaTeX via pandoc}}

\title{ENV 797 - Time Series Analysis for Energy and Environment
Applications \textbar{} Spring 2024}
\usepackage{etoolbox}
\makeatletter
\providecommand{\subtitle}[1]{% add subtitle to \maketitle
  \apptocmd{\@title}{\par {\large #1 \par}}{}{}
}
\makeatother
\subtitle{Assignment 6 - Due date 02/28/24}
\author{Student Name}
\date{}

\begin{document}
\maketitle

\begin{Shaded}
\begin{Highlighting}[]
\CommentTok{\#Load/install required package here}
\FunctionTok{library}\NormalTok{(ggplot2)}
\FunctionTok{library}\NormalTok{(forecast)}
\end{Highlighting}
\end{Shaded}

\begin{verbatim}
## Registered S3 method overwritten by 'quantmod':
##   method            from
##   as.zoo.data.frame zoo
\end{verbatim}

\begin{Shaded}
\begin{Highlighting}[]
\FunctionTok{library}\NormalTok{(tseries)}
\CommentTok{\#install.packages(sarima)}
\CommentTok{\#library(sarima)}
\FunctionTok{library}\NormalTok{(cowplot)}
\end{Highlighting}
\end{Shaded}

This assignment has general questions about ARIMA Models.

\hypertarget{q1}{%
\subsection{Q1}\label{q1}}

Describe the important characteristics of the sample autocorrelation
function (ACF) plot and the partial sample autocorrelation function
(PACF) plot for the following models:

\begin{itemize}
\tightlist
\item
  AR
\end{itemize}

\begin{quote}
Answer:For AR models the ACF will decay exponentially with time, and the
PACF will identify the order of the AR model.
\end{quote}

\begin{itemize}
\tightlist
\item
  MA
\end{itemize}

\begin{quote}
Answer: For MA models the ACF will identify the order of the MA model
and the PACF will decay exponentially.
\end{quote}

\hypertarget{q2}{%
\subsection{Q2}\label{q2}}

Recall that the non-seasonal ARIMA is described by three parameters
ARIMA where p is the order of the autoregressive component, d is the
number of times the series need to be differenced to obtain stationarity
and q is the order of the moving average component. If we don't need to
difference the series, we don't need to specify the ``I'' part and we
can use the short version, i.e., the ARMA

\begin{enumerate}
\def\labelenumi{(\alph{enumi})}
\tightlist
\item
  Consider three models: ARMA(1,0), ARMA(0,1) and ARMA(1,1) with
  parameters phi=0.6 and theta= 0.9. The phi refers to the AR
  coefficient and the theta refers to the MA coefficient. Use the
  \texttt{arima.sim()} function in R to generate n=100 observations from
  each of these three models. Then, using \texttt{autoplot()} plot the
  generated series in three separate graphs.
\end{enumerate}

\begin{Shaded}
\begin{Highlighting}[]
\NormalTok{phi }\OtherTok{\textless{}{-}} \FloatTok{0.6}
\NormalTok{theta }\OtherTok{\textless{}{-}} \FloatTok{0.9}
\NormalTok{n }\OtherTok{\textless{}{-}} \DecValTok{100}

\CommentTok{\# Generate data for ARMA(1,0)}
\NormalTok{arma\_10 }\OtherTok{\textless{}{-}} \FunctionTok{arima.sim}\NormalTok{(}\AttributeTok{model =} \FunctionTok{list}\NormalTok{(}\AttributeTok{order =} \FunctionTok{c}\NormalTok{(}\DecValTok{1}\NormalTok{, }\DecValTok{0}\NormalTok{, }\DecValTok{0}\NormalTok{), }\AttributeTok{ar =}\NormalTok{ phi), }\AttributeTok{n =}\NormalTok{ n)}

\CommentTok{\# Generate data for ARMA(0,1)}
\NormalTok{arma\_01 }\OtherTok{\textless{}{-}} \FunctionTok{arima.sim}\NormalTok{(}\AttributeTok{model =} \FunctionTok{list}\NormalTok{(}\AttributeTok{order =} \FunctionTok{c}\NormalTok{(}\DecValTok{0}\NormalTok{, }\DecValTok{0}\NormalTok{, }\DecValTok{1}\NormalTok{), }\AttributeTok{ma =}\NormalTok{ theta), }\AttributeTok{n =}\NormalTok{ n)}

\CommentTok{\# Generate data for ARMA(1,1)}
\NormalTok{arma\_11 }\OtherTok{\textless{}{-}} \FunctionTok{arima.sim}\NormalTok{(}\AttributeTok{model =} \FunctionTok{list}\NormalTok{(}\AttributeTok{order =} \FunctionTok{c}\NormalTok{(}\DecValTok{1}\NormalTok{, }\DecValTok{0}\NormalTok{, }\DecValTok{1}\NormalTok{), }\AttributeTok{ar =}\NormalTok{ phi, }\AttributeTok{ma =}\NormalTok{ theta), }\AttributeTok{n =}\NormalTok{ n)}

\CommentTok{\# Plot the generated series using autoplot()}
\FunctionTok{autoplot}\NormalTok{(arma\_10, }\AttributeTok{main =} \StringTok{"ARMA(1,0) Simulation"}\NormalTok{) }
\end{Highlighting}
\end{Shaded}

\includegraphics{Kuuskvere_C_TSA_A06_Sp24_files/figure-latex/unnamed-chunk-2-1.pdf}

\begin{Shaded}
\begin{Highlighting}[]
\FunctionTok{autoplot}\NormalTok{(arma\_01, }\AttributeTok{main =} \StringTok{"ARMA(0,1) Simulation"}\NormalTok{) }
\end{Highlighting}
\end{Shaded}

\includegraphics{Kuuskvere_C_TSA_A06_Sp24_files/figure-latex/unnamed-chunk-2-2.pdf}

\begin{Shaded}
\begin{Highlighting}[]
\FunctionTok{autoplot}\NormalTok{(arma\_11, }\AttributeTok{main =} \StringTok{"ARMA(1,1) Simulation"}\NormalTok{) }
\end{Highlighting}
\end{Shaded}

\includegraphics{Kuuskvere_C_TSA_A06_Sp24_files/figure-latex/unnamed-chunk-2-3.pdf}

\begin{enumerate}
\def\labelenumi{(\alph{enumi})}
\setcounter{enumi}{1}
\tightlist
\item
  Plot the sample ACF for each of these models in one window to
  facilitate comparison (Hint: use \texttt{cowplot::plot\_grid()}).
\end{enumerate}

\begin{Shaded}
\begin{Highlighting}[]
\CommentTok{\# Calculate and plot ACF for ARMA(1,0)}
\NormalTok{acf\_arma\_10 }\OtherTok{\textless{}{-}} \FunctionTok{acf}\NormalTok{(arma\_10, }\AttributeTok{main =} \StringTok{"ACF for ARMA(1,0)"}\NormalTok{)}
\end{Highlighting}
\end{Shaded}

\includegraphics{Kuuskvere_C_TSA_A06_Sp24_files/figure-latex/unnamed-chunk-3-1.pdf}

\begin{Shaded}
\begin{Highlighting}[]
\CommentTok{\# Calculate and plot ACF for ARMA(0,1)}
\NormalTok{acf\_arma\_01 }\OtherTok{\textless{}{-}} \FunctionTok{acf}\NormalTok{(arma\_01, }\AttributeTok{main =} \StringTok{"ACF for ARMA(0,1)"}\NormalTok{)}
\end{Highlighting}
\end{Shaded}

\includegraphics{Kuuskvere_C_TSA_A06_Sp24_files/figure-latex/unnamed-chunk-3-2.pdf}

\begin{Shaded}
\begin{Highlighting}[]
\CommentTok{\# Calculate and plot ACF for ARMA(1,1)}
\NormalTok{acf\_arma\_11 }\OtherTok{\textless{}{-}} \FunctionTok{acf}\NormalTok{(arma\_11, }\AttributeTok{main =} \StringTok{"ACF for ARMA(1,1)"}\NormalTok{)}
\end{Highlighting}
\end{Shaded}

\includegraphics{Kuuskvere_C_TSA_A06_Sp24_files/figure-latex/unnamed-chunk-3-3.pdf}

\begin{Shaded}
\begin{Highlighting}[]
\CommentTok{\# Create a grid of ACF plots using cowplot::plot\_grid()}
\CommentTok{\# cowplot::plot\_grid(acf\_arma\_10, acf\_arma\_01, acf\_arma\_11, ncol = 3)}

\CommentTok{\#plot\_grid(plotlist= autoplot(acf(arma\_10, main = "ACF for ARMA(1,0)"), autoplot(acf(arma\_01, main = "ACF for ARMA(0,1)"),autoplot(acf(arma\_11, main = "ACF for ARMA(1,1)")}
\end{Highlighting}
\end{Shaded}

\begin{enumerate}
\def\labelenumi{(\alph{enumi})}
\setcounter{enumi}{2}
\tightlist
\item
  Plot the sample PACF for each of these models in one window to
  facilitate comparison.
\end{enumerate}

\begin{Shaded}
\begin{Highlighting}[]
\CommentTok{\# Calculate and plot PACF for ARMA(1,0)}
\NormalTok{pacf\_arma\_10 }\OtherTok{\textless{}{-}} \FunctionTok{pacf}\NormalTok{(arma\_10, }\AttributeTok{main =} \StringTok{"PACF for ARMA(1,0)"}\NormalTok{)}
\end{Highlighting}
\end{Shaded}

\includegraphics{Kuuskvere_C_TSA_A06_Sp24_files/figure-latex/unnamed-chunk-4-1.pdf}

\begin{Shaded}
\begin{Highlighting}[]
\CommentTok{\# Calculate and plot PACF for ARMA(0,1)}
\NormalTok{pacf\_arma\_01 }\OtherTok{\textless{}{-}} \FunctionTok{pacf}\NormalTok{(arma\_01, }\AttributeTok{main =} \StringTok{"PACF for ARMA(0,1)"}\NormalTok{)}
\end{Highlighting}
\end{Shaded}

\includegraphics{Kuuskvere_C_TSA_A06_Sp24_files/figure-latex/unnamed-chunk-4-2.pdf}

\begin{Shaded}
\begin{Highlighting}[]
\CommentTok{\# Calculate and plot PACF for ARMA(1,1)}
\NormalTok{pacf\_arma\_11 }\OtherTok{\textless{}{-}} \FunctionTok{pacf}\NormalTok{(arma\_11, }\AttributeTok{main =} \StringTok{"PACF for ARMA(1,1)"}\NormalTok{)}
\end{Highlighting}
\end{Shaded}

\includegraphics{Kuuskvere_C_TSA_A06_Sp24_files/figure-latex/unnamed-chunk-4-3.pdf}

\begin{Shaded}
\begin{Highlighting}[]
\CommentTok{\# Create a grid of PACF plots using cowplot::plot\_grid()}
\CommentTok{\#pacf\_grid \textless{}{-} plot\_grid(pacf\_arma\_10, pacf\_arma\_01, pacf\_arma\_11, ncol = 3)}

\CommentTok{\# Display the grid}
\CommentTok{\#print(pacf\_grid)}
\end{Highlighting}
\end{Shaded}

\begin{enumerate}
\def\labelenumi{(\alph{enumi})}
\setcounter{enumi}{3}
\tightlist
\item
  Look at the ACFs and PACFs. Imagine you had these plots for a data set
  and you were asked to identify the model, i.e., is it AR, MA or ARMA
  and the order of each component. Would you be able identify them
  correctly? Explain your answer.
\end{enumerate}

\begin{quote}
Answer: The ARMA(1,0) model appears to have exponential decay in the ACF
and a significant spike at lag 1. This would Indicate an autoregressive
process with an order of 1. The ARMA 0,1 acf has a significant spike at
lag 1 and a pacf with exponential decay. This could be identified as a
moving average process with an order of 1. For the ARMA 1,1 model, it
has exponential decay in both its acf and pacf, with would make it
challenging to identify them correctly because it could indicate a a AR
or MA process with orders 1. However, oftentines interpreting these
graphs can be a challenge so I may not necessarily be able to identify
the model order correctly as if I did not know the information about the
models I expect to see.
\end{quote}

\begin{enumerate}
\def\labelenumi{(\alph{enumi})}
\setcounter{enumi}{4}
\tightlist
\item
  Compare the PACF values R computed with the values you provided for
  the lag 1 correlation coefficient, i.e., does phi=0.6 match what you
  see on PACF for ARMA(1,0), and ARMA(1,1)? Should they match?
\end{enumerate}

\begin{quote}
Answer: For ARMA(1,0), representing only an autoregressive (AR)
component, the PACF at lag 1 is expected to closely match phi = 0.6,
indicating the influence of the AR component.
\end{quote}

In ARMA(1,1), combining autoregressive (AR) and moving average (MA)
components, the PACF at lag 1 should also reflect phi = 0.6. However, at
lag 2, it reveals the combined effect of AR and MA coefficients. For
ARMA(0,1), with only a moving average (MA) component, the PACF at lag 1
is anticipated to align with the specified MA coefficient (theta = 0.9).
Overall, observed PACF values aligning with the specified AR and MA
coefficients validate that the simulated data adheres to the anticipated
autoregressive and moving average behaviors in the respective ARMA
models.

\begin{enumerate}
\def\labelenumi{(\alph{enumi})}
\setcounter{enumi}{5}
\tightlist
\item
  Increase number of observations to n=1000 and repeat parts (b)-(e).
\end{enumerate}

\begin{Shaded}
\begin{Highlighting}[]
\CommentTok{\# Increase number of observations}
\NormalTok{n }\OtherTok{\textless{}{-}} \DecValTok{1000}

\CommentTok{\# (b) Generate data for ARMA(1,0)}
\NormalTok{arma\_10\_large }\OtherTok{\textless{}{-}} \FunctionTok{arima.sim}\NormalTok{(}\AttributeTok{model =} \FunctionTok{list}\NormalTok{(}\AttributeTok{order =} \FunctionTok{c}\NormalTok{(}\DecValTok{1}\NormalTok{, }\DecValTok{0}\NormalTok{, }\DecValTok{0}\NormalTok{), }\AttributeTok{ar =}\NormalTok{ phi), }\AttributeTok{n =}\NormalTok{ n)}

\CommentTok{\# (c) Generate data for ARMA(0,1)}
\NormalTok{arma\_01\_large }\OtherTok{\textless{}{-}} \FunctionTok{arima.sim}\NormalTok{(}\AttributeTok{model =} \FunctionTok{list}\NormalTok{(}\AttributeTok{order =} \FunctionTok{c}\NormalTok{(}\DecValTok{0}\NormalTok{, }\DecValTok{0}\NormalTok{, }\DecValTok{1}\NormalTok{), }\AttributeTok{ma =}\NormalTok{ theta), }\AttributeTok{n =}\NormalTok{ n)}

\CommentTok{\# (d) Generate data for ARMA(1,1)}
\NormalTok{arma\_11\_large }\OtherTok{\textless{}{-}} \FunctionTok{arima.sim}\NormalTok{(}\AttributeTok{model =} \FunctionTok{list}\NormalTok{(}\AttributeTok{order =} \FunctionTok{c}\NormalTok{(}\DecValTok{1}\NormalTok{, }\DecValTok{0}\NormalTok{, }\DecValTok{1}\NormalTok{), }\AttributeTok{ar =}\NormalTok{ phi, }\AttributeTok{ma =}\NormalTok{ theta), }\AttributeTok{n =}\NormalTok{ n)}

\CommentTok{\# (e) PACF analysis}
\NormalTok{pacf\_arma\_10\_large }\OtherTok{\textless{}{-}} \FunctionTok{pacf}\NormalTok{(arma\_10\_large, }\AttributeTok{main =} \StringTok{"PACF for ARMA(1,0) {-} n=1000"}\NormalTok{)}
\end{Highlighting}
\end{Shaded}

\includegraphics{Kuuskvere_C_TSA_A06_Sp24_files/figure-latex/unnamed-chunk-5-1.pdf}

\begin{Shaded}
\begin{Highlighting}[]
\NormalTok{pacf\_arma\_01\_large }\OtherTok{\textless{}{-}} \FunctionTok{pacf}\NormalTok{(arma\_01\_large, }\AttributeTok{main =} \StringTok{"PACF for ARMA(0,1) {-} n=1000"}\NormalTok{)}
\end{Highlighting}
\end{Shaded}

\includegraphics{Kuuskvere_C_TSA_A06_Sp24_files/figure-latex/unnamed-chunk-5-2.pdf}

\begin{Shaded}
\begin{Highlighting}[]
\NormalTok{pacf\_arma\_11\_large }\OtherTok{\textless{}{-}} \FunctionTok{pacf}\NormalTok{(arma\_11\_large, }\AttributeTok{main =} \StringTok{"PACF for ARMA(1,1) {-} n=1000"}\NormalTok{)}
\end{Highlighting}
\end{Shaded}

\includegraphics{Kuuskvere_C_TSA_A06_Sp24_files/figure-latex/unnamed-chunk-5-3.pdf}

\begin{Shaded}
\begin{Highlighting}[]
\CommentTok{\# Display the plots}
\FunctionTok{print}\NormalTok{(pacf\_arma\_10\_large)}
\end{Highlighting}
\end{Shaded}

\begin{verbatim}
## 
## Partial autocorrelations of series 'arma_10_large', by lag
## 
##      1      2      3      4      5      6      7      8      9     10     11 
##  0.571 -0.005  0.040 -0.004  0.035 -0.024 -0.010  0.001  0.066 -0.025 -0.002 
##     12     13     14     15     16     17     18     19     20     21     22 
## -0.023  0.010  0.007 -0.002 -0.057  0.038 -0.032 -0.003 -0.002 -0.004  0.005 
##     23     24     25     26     27     28     29     30 
## -0.006  0.012 -0.033  0.008  0.061 -0.026  0.017 -0.036
\end{verbatim}

\begin{Shaded}
\begin{Highlighting}[]
\FunctionTok{print}\NormalTok{(pacf\_arma\_01\_large)}
\end{Highlighting}
\end{Shaded}

\begin{verbatim}
## 
## Partial autocorrelations of series 'arma_01_large', by lag
## 
##      1      2      3      4      5      6      7      8      9     10     11 
##  0.486 -0.306  0.210 -0.217  0.161 -0.145  0.110 -0.125  0.075 -0.139  0.064 
##     12     13     14     15     16     17     18     19     20     21     22 
## -0.102  0.035 -0.032  0.080 -0.004  0.064 -0.060  0.056 -0.031 -0.035 -0.044 
##     23     24     25     26     27     28     29     30 
## -0.028  0.013  0.024 -0.004  0.024  0.028  0.011 -0.033
\end{verbatim}

\begin{Shaded}
\begin{Highlighting}[]
\FunctionTok{print}\NormalTok{(pacf\_arma\_11\_large)}
\end{Highlighting}
\end{Shaded}

\begin{verbatim}
## 
## Partial autocorrelations of series 'arma_11_large', by lag
## 
##      1      2      3      4      5      6      7      8      9     10     11 
##  0.805 -0.450  0.320 -0.270  0.163 -0.100  0.099 -0.119  0.121 -0.135  0.099 
##     12     13     14     15     16     17     18     19     20     21     22 
## -0.047  0.076 -0.075  0.070  0.008 -0.020  0.055  0.034 -0.020 -0.015 -0.022 
##     23     24     25     26     27     28     29     30 
##  0.014 -0.012  0.022  0.032  0.000  0.042 -0.036 -0.022
\end{verbatim}

ARMA(1,0) - Autoregressive (AR) Model

ACF:

The ACF may show exponential decay, with decreasing correlations as the
lag increases. A significant spike at lag 1 indicates the presence of
the autoregressive term.

PACF:

The PACF at lag 1 should be close to the specified autoregressive
coefficient (phi = 0.6) due to the absence of a moving average
component.

ARMA(0,1) - Moving Average (MA) Model

ACF:

A significant spike at lag 1 indicates the presence of the moving
average term. Exponential decay in the ACF may be observed.

PACF:

The PACF should show a sharp cutoff after lag 1, representing the effect
of the moving average term. ARMA(1,1) - Autoregressive Moving Average
(ARMA) Model

ACF:

Exponential decay may be observed, indicating the presence of both AR
and MA components. A significant spike at lag 1 may indicate the
autoregressive term.

PACF:

The PACF at lag 1 reflects the autoregressive coefficient. The PACF at
lag 2 may indicate the combined effect of AR and MA coefficients.

In summary, the ACF and PACF plots for each model provide insights into
the presence and behavior of autoregressive and moving average
components, guiding the identification of the underlying ARMA structure
in the time series data.

\hypertarget{q3}{%
\subsection{Q3}\label{q3}}

\begin{enumerate}
\def\labelenumi{(\alph{enumi})}
\tightlist
\item
  Identify the model using the notation ARIMA notation i.e., identify
  the integers from the equation.
\end{enumerate}

p: AR determined by lag of highest dependent variable = 1 d:
differencing order to achieve stationarity-- 0 since there is no
differencing q: MA determined by highest lag of MA term, = 1

Seasonal: P: seasonal AR = 1, D: Seasonal differencing, D=0 since no
seasonal differencing Q: Seasonal MA = 0 since no seasonal moving avg
term

s: seasonal period, s=12

ARIMA(1, 0, 1)(1, 0, 0)12

\begin{enumerate}
\def\labelenumi{(\alph{enumi})}
\setcounter{enumi}{1}
\tightlist
\item
  Also from the equation what are the values of the parameters, i.e.,
  model coefficients.
\end{enumerate}

The coefficients in the equation represent the values of the parameters:

Autoregressive (AR) coefficients: 0.7

Moving Average (MA) coefficients: −0.1

Seasonal Autoregressive (SAR) coefficients: −0.25

Seasonal Moving Average (SMA) coefficients: none

Constant term: None.

Therefore, the values of the parameters are:

AR(1): 0.7 MA(1): -0.1 SAR(1): -0.25 In summary, the identified ARIMA
model is ARIMA(1, 0, 1)(1, 0, 0)12 with the corresponding parameter
values.

\hypertarget{q4}{%
\subsection{Q4}\label{q4}}

Simulate a seasonal ARIMA(0, 1) model with phi =0 .8 and theta = 0.5
using the \texttt{sim\_sarima()} function from package \texttt{sarima}.

\begin{Shaded}
\begin{Highlighting}[]
\NormalTok{phi }\OtherTok{\textless{}{-}} \FloatTok{0.8}
\NormalTok{theta }\OtherTok{\textless{}{-}} \FloatTok{0.5}

\CommentTok{\# Create SARIMA model}
\NormalTok{sarima\_model }\OtherTok{\textless{}{-}} \FunctionTok{list}\NormalTok{(}\AttributeTok{order =} \FunctionTok{c}\NormalTok{(}\DecValTok{0}\NormalTok{, }\DecValTok{1}\NormalTok{, }\DecValTok{0}\NormalTok{), }\AttributeTok{seasonal =} \FunctionTok{list}\NormalTok{(}\AttributeTok{order =} \FunctionTok{c}\NormalTok{(}\DecValTok{1}\NormalTok{, }\DecValTok{0}\NormalTok{, }\DecValTok{0}\NormalTok{), }\AttributeTok{period =} \DecValTok{12}\NormalTok{), }\AttributeTok{phi =}\NormalTok{ phi, }\AttributeTok{theta =}\NormalTok{ theta)}
\CommentTok{\#sarima\_model \textless{}{-} as.ts(sarima\_model)}

\CommentTok{\# Plot the generated series using autoplot()}
\CommentTok{\#autoplot(sarima\_model)}
\end{Highlighting}
\end{Shaded}

It would look seasonal if it had regular variations, but I cannot see
the plot due to an error. \#\# Q5

Plot ACF and PACF of the simulated series in Q4. Comment if the plots
are well representing the model you simulated, i.e., would you be able
to identify the order of both non-seasonal and seasonal components from
the plots? Explain.

\begin{Shaded}
\begin{Highlighting}[]
\CommentTok{\# Calculate and plot ACF and PACF for the simulated series with greater n}
\CommentTok{\#acf\_simulated\_large \textless{}{-} acf(sarima\_model, main = "ACF of Simulated Series (n=1000)")}
\CommentTok{\#pacf\_simulated\_large \textless{}{-} pacf(sarima\_model, main = "PACF of Simulated Series (n=1000)")}

\CommentTok{\# Display the plots}
\CommentTok{\#print(acf\_simulated\_large)}
\CommentTok{\#print(pacf\_simulated\_large)}
\end{Highlighting}
\end{Shaded}

Given the plots that should be outputted if I could get this to run
properly, I should be able to understand waht is simulated given the
decay of the lags and the impact at the seasonal intervals.

\end{document}
